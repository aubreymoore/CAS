% Commented out: 
% \addbibresource
% \includepdf

\documentclass[12pt,letterpaper,english,bibliography=totocnumbered, abstract=on]{scrartcl}

\usepackage{indentfirst}
\usepackage[titletoc]{appendix}
\usepackage[latin1]{inputenc}  % Enable direct input of special national characters
\usepackage[english]{babel}    % Language settings
\usepackage{lmodern}           % Enable Latin Modern fonts
\usepackage{color}
\usepackage{verbatim}
\usepackage[unicode=true,pdfusetitle,
bookmarks=true,bookmarksnumbered=false,bookmarksopen=false,
breaklinks=true,pdfborder={0 0 0},pdfborderstyle={},backref=false,colorlinks=true]
{hyperref}
\hypersetup{linkcolor=blue,citecolor=blue,urlcolor=blue}

\usepackage{booktabs}
\usepackage{multirow}
\usepackage{adjustbox}
\usepackage{threeparttable}
\usepackage[table]{xcolor}
\usepackage{csquotes}
\usepackage{soul} % for hiliting text: \hl

\usepackage[backend=biber, style=authoryear, maxbibnames=99, dashed=false]{biblatex}
\setlength\bibitemsep{2\itemsep}
\addbibresource{CASarticle.bib}
%\addbibresource{CRB.bib}

\usepackage{pdfpages}
\usepackage{float} % Allows use of H to place floats

\usepackage{pgfgantt}

\usepackage{framed}

% Prevent page breaks within paragraphs
% https://tex.stackexchange.com/questions/21983/how-to-avoid-page-breaks-inside-paragraphs
\widowpenalties 1 10000

\begin{document}

\title{Biological Control of the Cycad Aulacaspis Scale, \textit{Aulacaspis yasumatsui}}

\author{Contributions by Aubrey Moore}

\maketitle
%\footnote{\url{https://github.com/aubreymoore/2020-FS-CRB-biocontrol-project/blob/master/combined-proposal.pdf}}
\newpage
\tableofcontents

\pagebreak

\section{Abstract - Cave}

\section{Introduction - Cave}

\section{Economic impact of CAS - Cave and Wright}
\section{Ecological impact of CAS - Moore}

Ecological impact of CAS invasions is varies greatly with location, largely due to differences in characteristics of host plant host populations, climate, and presence of biological control agents.

When CAS arrived in Florida (1995) and Hawaii (1998), it became a pest of ornamental cycads which could be protected using a combination of pesticide applications and biological control. However, when CAS arrived in Guam (2003), it rapidly spread from ornamental cycads to the wild \textit{Cycas micronesica} population, causing an uncontrolled island-wide outbreak. At that time, \textit{C. micronesica} was the most abundant tree in Guam's forests \cite{donnegon_guams_2004}. 

In 2006, \textit{C. micronesica} was placed on the Red List of Threatend Species and in 2015 this plant was added to the US Endangered and Threatened Species List.

Cascading effects \cite{haynesExoticInvasivePest2005}

Effects on soil \cite{marlerTwoCycadSpecies2020}

\section{Natural enemies of CAS - Cave}

\section{Classical biological control}

\subsection{Florida - Cave}

\subsection{Hawaii - Wright}

\subsection{Guam - Moore}

\subsubsection{\textit{Rhyzobius lophanthae}}

About 100 adults of \textit{Rhyzobius lophanthae} were field collected on Maui and imported to Guam during November 2004. This coccinelid was originally 
introduced to California from Australia in 1892 and to Hawaii from California in 1894. It was observed feeding voraciously on CAS shortly after arrival of this new pest in Hawaii. 


\textit{R. lophanthae} was previously introduced to Guam on two separate occasions under various synonyms: \textit{R. satelles} Blackburn, \textit{Lindorus lophanthae} (Blaisdell), and \textit{R. pulchellus} Montrouzier (\cite{nafus_biological_1989}). 

In 1925 and 1926 Rhyzobius satelles was imported to Guam from California to control the coconut scale, \textit{Aspidiotus destructor} Signoret. However, attempts at field establishment failed.  

\cite{nafus_biological_1989} also report "In 1971, \textit{Rhyzobius satelles} Blackburn (as \textit{R. pulchellus} Montrouzier) was introduced to Guam from New Caledonia to aid in the control of coconut scales and citrus scales. A single specimen of \textit{R. satelles} was recovered in 1978, indicating establishment. The beetle, however, is very uncommon; an intensive survey of coconut insects in 1984 yielded no specimens."

The beetles from Maui were reared on scale-infested \textit{C. micronesica} cuttings placed in a large screened camping tent set up in a laboratory. Adult offspring were collected for field release by aspirating them from the walls of the tent into plasic vials. Field releases were initiated on February 16 2005 at the Guam National Wildlife Refuge at Ritidian Point. The beetles established readily. By July 7 2005 high densities on adults were observed on cycads anywhere within a 1 lm radius of the release site.  Establishment and dispersion of the beetles were monitored using yellow sticky traps deployed between June 2005 and May 2006. Unexpectedly, we were also able to monitor CAS crawlers and adult males using these traps (Fig. \ref{fig:sticky-traps})  (\cite{moore_biological_2017-2}). Following establishment of \textit{R. lophanthae} at Ritidian Point, laboratory-reared and field-collected beetles were released at about 30 other sites throughout Guam. 

\begin{figure}[H]
	\centering
	\includegraphics[width=0.7\linewidth]{sticky-traps}
	\caption{Caption goes here.}
	\label{fig:sticky-traps}
\end{figure}

By about 2010, \textit{R. lophanthae} larvae or adults could be found on almost every CAS-infested cycad on Guam, preventing CAS from killing mature cycads. By 2010, about 90\% of wild cycads had been killed on Guam (REF). Unfortunately, the \textit{C. micronesica} population is not recovering because almost all seeds and seedlings are being killed by CAS and other causes (REF). \cite{marlerVerticalStratificationPredation2013} showed that \textit{R. lophanthae} predation of CAS is significantly reduced close to the ground and suggest that this may account for failed biocontrol of CAS on seedlings. They also suggest:
\begin{displayquote}
The causes of reduced scale predation by
\textit{R. lophanthae} near the ground are unknown,
but a parasitoid biological control agent may
not exhibit these same limitations. Furthermore, because a parasitoid would be much
smaller than \textit{R. lophanthae}, it would likely be
better able to access scale infestations within
cracks and crevices on \textit{C. micronesica} and
\textit{C. revoluta} trees.
\end{displayquote}

\subsubsection{Other biological control agents}  

Several attempts at introducing CAS parasitoids to Guam have failed.  
  

Ask Mark, Janis about Bernarr's report on fortuitous introduction of CAS parasitoids.

Ask Reddy about his report.

Ask Arnold Harra.


\subsection{Elsewhere - Cave}

\section{Prospects for future action - Cave, Wright, and Moore}

\section{Miscellaneous Notes-Moore}

THESE NOTES SHOULD PROBABLY BE INCLUDED IN OTHER SECTIONS.

\paragraph{Invasive pathway for transport to Guam} Arrival on of CAS on Guam was predicted. On February 13 2000 T. E. Marler published an article in the Gardening section of the Pacific Daily News entitled \textit{Looking out for scale insects} (\cite{haynesExoticInvasivePest2005}). This article warned of the pending arrival of CAS on Guam and pleaded for a stop to cycad imports. CAS was first detected in Tumon Beach, Guam near the end of 2003 on \textit{C. micronesica} and \textit{C. revoluta} growing growing as ornamental plants at two hotels. In those days, almost every hotel had cycad displays near their entrances. Infested imported cycads is the presumed pathway for introduction of CAS to Guam. However, there is little evidence available. There are no records of legal cycad importation in the two years prior to detection of CAS on Guam (R. Campbell, Guam Plant Inspection Facility, personal communication).  

An intriguing possibility is that CAS arrived on Guam as crawler. For many years there was an active infestation of CAS on \textit{C. revoluta} growing in an outdoor garden at the Honolulu International Airport located within a few hundred meters of where passengers boarded a daily flight to Guam. Possibly crawlers were carried on clothing of passengers visiting this garden or airborne and survived the 7.5 hour flight to Guam. Alternatively, airborne crawlers me have been blown into cargo holds or other spaces on the aircraft.

\newpage
\printbibliography[heading=bibintoc]

\end{document}
