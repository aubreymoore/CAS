
\begin{appendices}
	
	Notes from key references.

\section{Haynes and Marler 2005}

\subsection{fadang plants provide crucial food for other organisms}
\begin{displaycquote}{haynesExoticInvasivePest2005}
A second reason for funding this project is that fadang plants provide crucial food for other organisms.
The fleshy, aromatic covering of fadang seeds is a preferred food item for the endangered Mariana
fruit bat, Pteropus marianus marianus. Fadang is so resistant to most types of disturbance that its
seeds are sometimes the only bat food item available in the forest following the destructive winds of a
passing cyclone. Fewer than 100 Mariana fruit bats remain on Guam and it is unknown what effect the
loss of fadang will have on these endangered bats. (For more information on Guam’s fruit bats, refer
to the following website: http://www.fws.gov/pacific/pacificislands/wesa/marianabatindex.html.)
We have only just begun our herbivory surveys of Cycas micronesica. In our preliminary work, we
have identified the indigenous stem borer, Dihammus marianarum (Coleoptera: Cerambycidae), as a
common cycad consumer. When these surveys are completed, there may be other native arthropod
cycad consumers. Thus, the loss of this host species may also lead to the loss of several native species.
\end{displaycquote}

\subsection{symbiotic relationships}
\begin{displaycquote}{haynesExoticInvasivePest2005}
A third reason this project deserves emergency status is that cycad plants are part of a tripartite
symbiotic system of organisms. As with many species of plants, cycad roots are invaded by
mycorrhizal fungi. However, cycads also produce coralloid roots that are host to nitrogen-fixing
cyanobacteria. Work in ex-situ sites reveals that any available genotype of cyanobacteria can invade
coralloid roots. This is also probably true for mycorrhizae. Thus, determining the genetic variation of
cycad mycorrhizae and cyanobacteria in the native habitat is needed to shed light on conservation of
these important organisms. Loss of Cycas micronesica from Guam’s various populations could easily
result in the permanent loss of local genotypes of these other organisms that make up this complex,
interconnected system.
\end{displaycquote}

\subsection{pollinator(s)}
\begin{displaycquote}{haynesExoticInvasivePest2005}
A fourth reason for considering this a top priority for emergency funding is the nature of cycad
pollination dynamics. We now know that cycads are pollinated almost exclusively by obligate insect
pollinators (Norstog \& Nicholls, 1997; Whitelock, 2002). We are attempting to identify the insect(s)
which co-evolved with Guam’s cycad population as the pollinators. At the present time, so many of
the habitats are infested with CAS that it is difficult to find individual cycads suitable for pollination
studies. The probability of one or more endemic, obligate insect pollinators occurring on Guam is
highly likely. These beneficial insects will also be lost along with Guam’s cycad population should we
continue to stand by and let the CAS epidemic continue unchecked.
\end{displaycquote}

\subsection{fadang is an iconic plant}
\begin{displaycquote}{haynesExoticInvasivePest2005}
Our height increment data indicate that many of Guam’s coastal cycad plants are hundreds of years
old. These plants have survived the Spanish-American War and two world wars; they have survived
innumerable tropical cyclones; they have survived the invasion of intentionally introduced feral deer
and pig populations and the accidental introduction of various insect species. Some of these individual
plants “watched” Ferdinand Magellan and his fleet sail along Guam’s coast on 6 March 1521. And
they endured the Spanish-Chamorro Wars that decimated the indigenous human population. Truly, the
remaining plants comprise a long-lived botanical and cultural treasure, one that is in danger of
disappearing forever. We simply cannot elect to continue to watch this tragedy without attempting to
intervene. The impending cascade of detrimental effects is looming too large to justify apathy.
\end{displaycquote}

\section{2005 - Report and recommendations on cycad aulacaspis scale}

\subsection{list of known CAS biocontrol agents}
\begin{displaycquote}{tangReportRecommendationsCycad2005a}
BIOLOGICAL CONTROL

In practice, the introduction of predators or parasitoids is the most cost- and labor-effective method of
controlling scale insect infestations. It is also the standard approach for long-term control of introduced
exotic scale pests (Meyerdirk, 2002). The existence of effective natural biocontrol agents for CAS can be
assumed, since the scale is usually in low or moderate densities in wild populations of Cycas in Thailand,
where CAS is native, where it does not cover foliage like it does in cultivated plants (Tang et al., 1997).
At least three such predators or parasitoids have been identified and tested in the field as biocontrol agents
for CAS.
In 1996, Dr. Richard Baranowski of the University of Florida-Homestead (retired), working with Banpot
Naponpeth, director of the Natural Biological Control Research Center at Kawetsart University in
Bangkok, Thailand, identified two potential biocontrol organisms. This research was conducted in part on
the grounds of Nong Nooch Tropical Garden. Both insects were evaluated and then widely released in
Florida as biocontrol agents of CAS.
Coccobius fulvus (Compere \& Annecke) (Hymenoptera: Aphelinidae) is a parasitoid 3 wasp no larger than its
host, ca. 1 mm long (see Fig. 1). Observations of this organism in south Florida suggest that this wasp, by itself,
is not aggressive enough to control CAS on Cycas plants, such as C. revoluta, that are highly susceptible to
CAS (Caldwell, 2005), but it can be effective in large, heavily infested plants of other Cycas species and/or
plants with dense foliage in which pesticide application is inhibited (Wiese et al., in press). Coccobius fulvus
has also been released as a biocontrol agent of other diaspid scale insects in the U.S. (Meyerdirk, 2002).
Cybocephalus binotatus Grouvelle (Coleoptera: Nitidulidae) is a predatory beetle not much longer than CAS
(see Fig. 2). The adult punctures the scale cover and chews on the living scale underneath; it will also deposit
eggs under the scale cover, where its larvae then feed on CAS eggs. A study of the effects of this predator on a
similar species of Aulacaspis on mangos suggests that, because it requires a substantial scale population to
maintain effective numbers, it must be re-released periodically into infested areas to maintain effective control
(Lagadec, 2004). Thus, an active, ongoing release program may be necessary for effective control using this
biocontrol agent. Recently the beetle released in Florida has been re-identified as Cybocephalus nipponicus
(Endrody-Younga) (R. Cave, pers. comm.).
The ladybird beetle, Rhyzobius lophanthae (Blaisdell) (Coleoptera: Coccinellidae), a native of Australia
that is often called the “scale destroyer,” has been successfully used as a control agent for CAS on
cultivated Cycas plants in Hawaii by the University of Hawaii and the Hawaii Department of Agriculture
(Hara et al., undated). It has also been reared and released on Guam since February 2005 to combat the
CAS infestation in wild Cycas micronesica populations. This beetle seems to be taking hold and
spreading on Guam, but effective control of CAS has not yet been achieved to date (A. Brooke \& I. Terry,
pers. comm.). Although this beetle was established in Florida as a predator of other scale insects prior to
the outbreak of CAS, it has not been observed to have any significant impact on CAS infestations (R.
Cave, pers. comm.). To be effective, this predator must be reared and released in significant numbers in
infested areas and re-released as outbreaks reoccur.
Other potential biocontrol agents have been identified or suggested, but these require more surveys in the
wild, in addition to subsequent lab and field evaluations, before their effectiveness will be known. They
include insects, mites, and fungi.
Insects
The twice stabbed lady beetle, Chilocorus stigma (Say), a species native to the U.S., has been observed
feeding on CAS on cultivated Cycas in Florida (Cave \& Duetting, 2004; Tang \& Skarlinsky, unpubl.).
Again, observations suggest that, to be effective, such coccinellid beetles require repeated mass release in
infected areas. This species is a generalist that attacks a variety of diaspids.
The following parasitoids of the wasp families Aphelinidae and Encyrtidae have been identified in
southern China and Vietnam and may have potential as biocontrol agents of CAS (Cave, unpubl.;
Meyerdirk, 2002):
•
•
3
Aprostocetus sp. (possibly A. purpuratus
Girault)
Arrhenophagus sp. (possibly A. chionaspidis
Aurivillius)
•
•
•
•
Aphytis lepidosaphes Compere
Encarsia sp.
Pteroptrix chinensis (Howard)
Thomsonisca sankarani Subba Rao

Mites
The mite, Hemiarcoptes sp. (prob. H. coccophagus), has been found to control a related species of
Aulacaspis in a lab setting (Meyerdirk, 2002).
Fungi
An unidentified fungus is known to grow on masses of the scale, Aulacaspis tegalensis (Zehntner)
(Meyerdirk, 2002). Another unidentified fungus has been observed growing on CAS in Florida (Caldwell,
2005).
\end{displaycquote}

\subsection{CAS is fatal}

\begin{displaycquote}{tangReportRecommendationsCycad2005a}
CAS will kill an infected Cycas host plant within a year if no control measures
are taken.
\end{displaycquote}

\subsection{Primary Recommendations to the CSG}
\begin{displaycquote}{tangReportRecommendationsCycad2005a}
To reiterate, the activities of major urgency that the CSG should pursue are the following:

1. A major priority must be to promote research on identifying new biocontrol agents for CAS and
determining how to improve the effectiveness and accelerate the establishment of biocontrol organisms
in newly infested areas.

2. Work together with the IUCN/SSC Invasive Species Specialist Group and the Global Invasive Species
Programme to alert plant protection organizations of countries throughout the tropics and subtropics—
especially those that possess wild species of Cycas—about the threat of CAS. Provide them with
information and techniques for effective exclusion of CAS. This will require tapping into the IUCN
and/or other high profile media outlets.

3. Assist with locating funding for current control efforts in Guam and Taiwan. Aid in collating and
documenting such efforts, so as to identify the most effective techniques and avoid repeated
duplication of ineffective control measures.
\end{displaycquote}

\end{appendices}

